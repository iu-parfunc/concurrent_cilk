

\section{A Simple Program}\label{s:prog_simple}

Consider the the program in figure \ref{fig:prog1_serial}: it enters at main, grows the stack by
three C stack frames before returning to main and exiting. 

\begin{figure}[H]
\centering
\framebox[4cm]{main()}\\
\framebox[4cm]{\parbox {4cm}{ \centering foo() \\ locals: x, y, z }}\\
\framebox[4cm]{add()}\\
\caption{Call Stack of Figure \ref{fig:prog1_serial}.}
\label{fig:prog1_stack_serial}
\end{figure}


\begin{figure}[H]
\centering
\begin{lstlisting}
#include <stdio.h>

int add(int x, int y)
{
  return x + y;
}

void foo(void)
{
  int x;
  int y;
  int z;

  x = add(5, 6);
  y = add(7, 8);

  z = add(x, y);

  printf("result: %d\n", z);
}

int main() 
{ 
  foo();
  return 0;
}
\end{lstlisting}
\caption{A Simple Serial Program}
\label{fig:prog1_serial}
\end{figure}

In contrast, figure \ref{fig:prog1_parallel} shows how the same code as figure
\ref{fig:prog1_serial} is parallelized using Cilk. The 

\begin{figure}[H]
\centering
\begin{lstlisting}
#include <stdio.h>

int add(int x, int y)
{
  return x + y;
}

void foo(void)
{
  int x;
  int y;
  int z;

  x = cilk_spawn add(5, 6);
  y = cilk_spawn add(7, 8);
  cilk_sync;

  z = add(x, y);
  printf("result: %d\n", z);
}

int main() 
{ 
  foo();
  return 0;
}
\end{lstlisting}
\caption{Parallelized With Cilk}
\label{fig:prog1_parallel}
\end{figure}



